\documentclass[]{book}
\usepackage{lmodern}
\usepackage{amssymb,amsmath}
\usepackage{ifxetex,ifluatex}
\usepackage{fixltx2e} % provides \textsubscript
\ifnum 0\ifxetex 1\fi\ifluatex 1\fi=0 % if pdftex
  \usepackage[T1]{fontenc}
  \usepackage[utf8]{inputenc}
\else % if luatex or xelatex
  \ifxetex
    \usepackage{mathspec}
  \else
    \usepackage{fontspec}
  \fi
  \defaultfontfeatures{Ligatures=TeX,Scale=MatchLowercase}
\fi
% use upquote if available, for straight quotes in verbatim environments
\IfFileExists{upquote.sty}{\usepackage{upquote}}{}
% use microtype if available
\IfFileExists{microtype.sty}{%
\usepackage{microtype}
\UseMicrotypeSet[protrusion]{basicmath} % disable protrusion for tt fonts
}{}
\usepackage[margin=1in]{geometry}
\usepackage{hyperref}
\hypersetup{unicode=true,
            pdftitle={MTXQCvX2 documentation},
            pdfauthor={Christin Zasada},
            pdfborder={0 0 0},
            breaklinks=true}
\urlstyle{same}  % don't use monospace font for urls
\usepackage{natbib}
\bibliographystyle{apalike}
\usepackage{color}
\usepackage{fancyvrb}
\newcommand{\VerbBar}{|}
\newcommand{\VERB}{\Verb[commandchars=\\\{\}]}
\DefineVerbatimEnvironment{Highlighting}{Verbatim}{commandchars=\\\{\}}
% Add ',fontsize=\small' for more characters per line
\usepackage{framed}
\definecolor{shadecolor}{RGB}{248,248,248}
\newenvironment{Shaded}{\begin{snugshade}}{\end{snugshade}}
\newcommand{\KeywordTok}[1]{\textcolor[rgb]{0.13,0.29,0.53}{\textbf{#1}}}
\newcommand{\DataTypeTok}[1]{\textcolor[rgb]{0.13,0.29,0.53}{#1}}
\newcommand{\DecValTok}[1]{\textcolor[rgb]{0.00,0.00,0.81}{#1}}
\newcommand{\BaseNTok}[1]{\textcolor[rgb]{0.00,0.00,0.81}{#1}}
\newcommand{\FloatTok}[1]{\textcolor[rgb]{0.00,0.00,0.81}{#1}}
\newcommand{\ConstantTok}[1]{\textcolor[rgb]{0.00,0.00,0.00}{#1}}
\newcommand{\CharTok}[1]{\textcolor[rgb]{0.31,0.60,0.02}{#1}}
\newcommand{\SpecialCharTok}[1]{\textcolor[rgb]{0.00,0.00,0.00}{#1}}
\newcommand{\StringTok}[1]{\textcolor[rgb]{0.31,0.60,0.02}{#1}}
\newcommand{\VerbatimStringTok}[1]{\textcolor[rgb]{0.31,0.60,0.02}{#1}}
\newcommand{\SpecialStringTok}[1]{\textcolor[rgb]{0.31,0.60,0.02}{#1}}
\newcommand{\ImportTok}[1]{#1}
\newcommand{\CommentTok}[1]{\textcolor[rgb]{0.56,0.35,0.01}{\textit{#1}}}
\newcommand{\DocumentationTok}[1]{\textcolor[rgb]{0.56,0.35,0.01}{\textbf{\textit{#1}}}}
\newcommand{\AnnotationTok}[1]{\textcolor[rgb]{0.56,0.35,0.01}{\textbf{\textit{#1}}}}
\newcommand{\CommentVarTok}[1]{\textcolor[rgb]{0.56,0.35,0.01}{\textbf{\textit{#1}}}}
\newcommand{\OtherTok}[1]{\textcolor[rgb]{0.56,0.35,0.01}{#1}}
\newcommand{\FunctionTok}[1]{\textcolor[rgb]{0.00,0.00,0.00}{#1}}
\newcommand{\VariableTok}[1]{\textcolor[rgb]{0.00,0.00,0.00}{#1}}
\newcommand{\ControlFlowTok}[1]{\textcolor[rgb]{0.13,0.29,0.53}{\textbf{#1}}}
\newcommand{\OperatorTok}[1]{\textcolor[rgb]{0.81,0.36,0.00}{\textbf{#1}}}
\newcommand{\BuiltInTok}[1]{#1}
\newcommand{\ExtensionTok}[1]{#1}
\newcommand{\PreprocessorTok}[1]{\textcolor[rgb]{0.56,0.35,0.01}{\textit{#1}}}
\newcommand{\AttributeTok}[1]{\textcolor[rgb]{0.77,0.63,0.00}{#1}}
\newcommand{\RegionMarkerTok}[1]{#1}
\newcommand{\InformationTok}[1]{\textcolor[rgb]{0.56,0.35,0.01}{\textbf{\textit{#1}}}}
\newcommand{\WarningTok}[1]{\textcolor[rgb]{0.56,0.35,0.01}{\textbf{\textit{#1}}}}
\newcommand{\AlertTok}[1]{\textcolor[rgb]{0.94,0.16,0.16}{#1}}
\newcommand{\ErrorTok}[1]{\textcolor[rgb]{0.64,0.00,0.00}{\textbf{#1}}}
\newcommand{\NormalTok}[1]{#1}
\usepackage{longtable,booktabs}
\usepackage{graphicx,grffile}
\makeatletter
\def\maxwidth{\ifdim\Gin@nat@width>\linewidth\linewidth\else\Gin@nat@width\fi}
\def\maxheight{\ifdim\Gin@nat@height>\textheight\textheight\else\Gin@nat@height\fi}
\makeatother
% Scale images if necessary, so that they will not overflow the page
% margins by default, and it is still possible to overwrite the defaults
% using explicit options in \includegraphics[width, height, ...]{}
\setkeys{Gin}{width=\maxwidth,height=\maxheight,keepaspectratio}
\IfFileExists{parskip.sty}{%
\usepackage{parskip}
}{% else
\setlength{\parindent}{0pt}
\setlength{\parskip}{6pt plus 2pt minus 1pt}
}
\setlength{\emergencystretch}{3em}  % prevent overfull lines
\providecommand{\tightlist}{%
  \setlength{\itemsep}{0pt}\setlength{\parskip}{0pt}}
\setcounter{secnumdepth}{5}
% Redefines (sub)paragraphs to behave more like sections
\ifx\paragraph\undefined\else
\let\oldparagraph\paragraph
\renewcommand{\paragraph}[1]{\oldparagraph{#1}\mbox{}}
\fi
\ifx\subparagraph\undefined\else
\let\oldsubparagraph\subparagraph
\renewcommand{\subparagraph}[1]{\oldsubparagraph{#1}\mbox{}}
\fi

%%% Use protect on footnotes to avoid problems with footnotes in titles
\let\rmarkdownfootnote\footnote%
\def\footnote{\protect\rmarkdownfootnote}

%%% Change title format to be more compact
\usepackage{titling}

% Create subtitle command for use in maketitle
\newcommand{\subtitle}[1]{
  \posttitle{
    \begin{center}\large#1\end{center}
    }
}

\setlength{\droptitle}{-2em}

  \title{MTXQCvX2 documentation}
    \pretitle{\vspace{\droptitle}\centering\huge}
  \posttitle{\par}
    \author{Christin Zasada}
    \preauthor{\centering\large\emph}
  \postauthor{\par}
      \predate{\centering\large\emph}
  \postdate{\par}
    \date{2018-10-26}

\usepackage{booktabs}

\usepackage{amsthm}
\newtheorem{theorem}{Theorem}[chapter]
\newtheorem{lemma}{Lemma}[chapter]
\theoremstyle{definition}
\newtheorem{definition}{Definition}[chapter]
\newtheorem{corollary}{Corollary}[chapter]
\newtheorem{proposition}{Proposition}[chapter]
\theoremstyle{definition}
\newtheorem{example}{Example}[chapter]
\theoremstyle{definition}
\newtheorem{exercise}{Exercise}[chapter]
\theoremstyle{remark}
\newtheorem*{remark}{Remark}
\newtheorem*{solution}{Solution}
\begin{document}
\maketitle

{
\setcounter{tocdepth}{1}
\tableofcontents
}
\chapter{Welcome}\label{welcome}

This documentation introduced to you how to use MTXQCvX2 in order to
assess the quality of your GC-MS derived data, perform the determination
of calibration curves and absolute quantification. It furthermore
provides you two normalisation strategies and the calculation of
quantities in, e.g., pmol/1e+6 cells or pmol/mg tissue.

MTXQCvX2 does also enable the calculation of stable isotopic
incorporation and the evaluation of the underlying data, the mass
isotopomer distributions (MIDs).

The tool has been set up to support the in-lab developed workflow for
quantitative metabolomics experiments using the in-house developed
software Maui for the annotation of data. MTXQCvX2 bridges the gap
between quality control and first data post-processing / analysis of
GC-MS derived data (MTXQCvX2\_part1, MTXQCvX2\_part2).

Nevertheless MTXQCvX2 includes a module in order to integrate all kind
of data provided in spreadsheet-format, e.g., derived from metmax,
extracting required information and creating corresponding files
(MTXQCvX2\_part4).

Both workflows are introduced in the distinct chapters including their
required input parameter (chapter \ref{workflows}). Technical relevant
information are summarised in chapter \ref{tables}.

\chapter{Introduction}\label{intro}

Experimental and mathematical concepts have been introduced for the
pulsed stable isotope resolved metabolomics (pSIRM) approach in
\citep{Pietzke2014}.

\chapter{Workflow for Maui-annotation proejcts}\label{maui}

\section{Read this in case}\label{read-this-in-case}

\begin{itemize}
\tightlist
\item
  you have run a Maui project
\item
  exported all required container (see \ref{container})
\item
  you have a copy of sequence list and experimental conditions
\item
  you know the extraction procedure
\end{itemize}

The following article describes briefly how to use MTXQCvX2 in case you
used Maui for the annotation of your metabolomics project. It does not
matter if you have performed an experiment including stable isotopes or
if you just aim for the quantification of a few intermediates.

\section{Quick view}\label{quick-view}

\begin{enumerate}
\def\labelenumi{\arabic{enumi}.}
\tightlist
\item
  Setup a new R-project and copy MTXQC template files and folders
\item
  Knit with parameter: \texttt{MTXQC\_init.Rmd} and create project
  folder, e.g., \texttt{psirm\_glucose}
\item
  Copy input files and rename \texttt{ManualQuantTable.tsv}
  (e18205cz.tsv)
\item
  Create \texttt{annotation.csv} and \texttt{sample\_extracts.csv} files
\item
  Define the internal extraction standard
\item
  Knit with parameter: \texttt{MTXQC\_ExperimentalSetup.Rmd}
\item
  Knit with parameter: \texttt{MTXQC\_part1.Rmd}
\item
  Knit with parameter: \texttt{MTXQC\_part2.Rmd}
\item
  If required, proceed with \texttt{MTXQC\_part3.Rmd} for
  ManualValidation
\end{enumerate}

\section{Input files}\label{input-files}

Three different kind of export functions have been implemented in Maui.
These functions provide the export of the actual data into \texttt{.csv}
or \texttt{.tsv} files that are directly usable as input files for
MTXQCvX2. Please refer to section \ref{mauiexport} how you perform the
export and which containers have to be exported using what export
function and where to copy them in \texttt{psirm\_glucose/input/}.

Certain circumstances might wish you to combine \emph{multiple
MAUI-projects} into one MTXQC-project. This might be the case when you
run the same samples in split and splitless mode on the machine or your
experimental setup has been measured in multiple batches in order to
avoid derivatisation effects.

It is highly recommended to combine the input files derived from a
different Maui projects beforehand the analysis. In that way you have
only to work with a single file
\texttt{CalculationFileData.csv}\footnote{stored in
  \texttt{psirm\_glucose/output/quant/...}} containing all data of the
your experiment.

The herein described process provides a quick way how to combine the
exported files from different Maui projects. The script
\texttt{combine-sets.R}\footnote{inst/template\_files/\ldots{}}
automatically saves all combined files into the correct \texttt{input}
folder. Just update the folder and subfolder names. All the rest has
been taken care of for you.

\begin{enumerate}
\def\labelenumi{\arabic{enumi}.}
\tightlist
\item
  Create in the MTXQC-project folder (e.g., \texttt{psirm\_glucose/}) a
  new folder called \texttt{raw-data}
\item
  Create a subfolder for each Maui-project in
  \texttt{psirm\_glucose/raw\_data/...}
\item
  Copy into this folder all your Maui-derived input files altogether
\item
  Update the parameter of \texttt{combine-sets.R}, meaning folder name
  definitions, file
\item
  Execute the R script
\item
  Merged files have been generated and copied into the corresponding
  folder: \texttt{psirm\_glucose/input-folder/gc/...} or
  \texttt{psirm\_glucose/input-folder/inc/...}
\item
  Copy the renamed \texttt{ManualQuantTable.tsv} files of each Maui
  project into \texttt{psirm\_glucose/input/quant/...}
\end{enumerate}

\section{Annotation-file}\label{annotation-file}

The annotation file relate file names with experimental conditions or
specify quantification standards in your batch. Two columns -
\textbf{File and Type} - are obligatory and have to be present in the
annotation file. In the case of their absence
\texttt{MTXQCvX\_part1.Rmd} stops processing and shows an error message.

A quick way to generate an annotation file is described below:

\begin{enumerate}
\def\labelenumi{\arabic{enumi}.}
\tightlist
\item
  Copy the first row / header of \texttt{quantMassAreaMatrix.csv} file
\item
  Paste \& transpose the content into a new Excel-File into column A
\item
  Change the first entry: Metabolite -\textgreater{} File
\item
  Remove the entry QuantMasses at the very end of the column A
\item
  Add the column Type and specify each file either as \textbf{sample} or
  \textbf{addQ1\_dilution}\footnote{see for further details
    additionalQuant}
\item
  Add more columns specifying your experimental conditions, e.g.,
  Cellline and Treatment\footnote{optimal: two-three parameter, max:
    four parameter. Consider possible combinations, e.g.,
    HCT116-control, HCT116-BPTES}
\item
  Save the content as \texttt{csv-file} in the
  \texttt{psirm\_glucose/input/...}
\end{enumerate}

\section{Sample\_extracts-file}\label{sample_extracts-file}

The \texttt{sample\_extracts.csv} file is required in order to determine
automatically absolute quantities in the manner of pmol/1e+6 cells or
pmol/mg tissue in the \texttt{CalculationFileData.csv}.

This file requires two obligatory columns - \textbf{Extract\_vol} and
\textbf{Unit}\footnote{Define: count, mg or ul}. Please specify for each
experimental condition the amount of extracted cells (count), tissue
(mg) or volume of blood/plasme (ul) in the unit shown in the brackets.\\
The names of the columns of the experimental conditions have to match up
with the annotation file. Save the file in the folder
\texttt{psirm\_glucose/input/...}.

If the defined experimental conditions do not match up with the
annotation \texttt{MTXQCvX2\_part1.Rmd} exit data processing. A template
file is saved for review and usage at \texttt{inst/template\_files/...}

\section{Internal Standard}\label{internal-standard}

MTXQCvX2 allows the specification of project-specific internal
extraction standards. The only thing you need to do is to define the
corresponding compounds as an internal standard in the
\texttt{conversion\_metabolite.csv} file. To do so, add
\texttt{InternalStandard} to the compound in last column
\texttt{Standard}.

For an classical pSIRM experiment in the Kempa lab we are using cinnamic
acid. The evaluation of this compound has been integrated in Maui. Peak
areas of cinnamic acid are exported from a distinct container called
\texttt{cinAcid}. The exported file has to be renamed to
\texttt{InternalStandard.csv} though and moved to
\texttt{psirm\_glucose/input/gc/...}.

If you have used a different compound as an internal extraction standard
you might need to extract the peak areas of this compound from the Maui
export file \texttt{quantPeakAreasMatrix.csv} file and save it in the
folder \texttt{psirm\_glucose/input/gc/InternalStandard.csv},
respectively. Prerequisite - you have annotated the compound in Maui.

The report of \texttt{MTXQCvX2\_part1.Rmd} includes the defined internal
standard for each project in a message.

\chapter{Workflow for Metmax-extracted projects}\label{metmax}

\section{You want to follow this
\ldots{}}\label{you-want-to-follow-this}

\begin{itemize}
\tightlist
\item
  in case you have measured samples and quantification standards by
  GC-MS
\item
  performed the annotation of intermediates in ChromaToF or vendor
  software
\item
  exported all information into \texttt{.txt} files
\item
  used metmax to extract peak areas / mass isotopomer distributions
  (MIDs)
\end{itemize}

\section{Introduction}\label{introduction}

This document describes how to use MTXQCvX2 in combination with
metmax\footnote{\url{http://gmd.mpimp-golm.mpg.de/apps/metmax/default.htm}}.

Historically, MTXQCvX2 has been developed and optimized for Maui-derived
input files. The \texttt{MTXQCvX2-part4.Rmd} functions as a converter of
metmax-derived files in order to create suitable input formats for
\texttt{MTXQCvX-part1.Rmd}.

This module could also be used to convert tables derived from other
programs as long as they are stick with the herein described table
formats. Mandatory columns are referenced in the text for each kind of
input file.

The general workflow of the NMTXQCvX2 project is briefly shown below in
quick view. More detailed instructions are summarised in the following
paragraphs.

For more detailed explanations about the individual input parameter for
each module of MTXQCvX2 please proceed to read the documentation about
the individual modules and their knitting parameter. The relation of
knitting parameter, input and output files are described in each
section.

\section{Quick view}\label{quick-view-1}

\begin{enumerate}
\def\labelenumi{\arabic{enumi}.}
\tightlist
\item
  Generate input files: run \texttt{MTXQC\_part4.Rmd}\footnote{read here
    the instructions}
\item
  Setup R-project and copy MTXQC-files
\item
  Knit with parameter: \texttt{MTXQC\_init.Rmd}
\item
  Copy input files into corresponding folders
\item
  Create annotation.csv and sample\_extracts.csv files\footnote{Details
    further down this document}
\item
  Update metabolite names in
  \texttt{conversion\_metabolite.csv}\footnote{Column:
    Metabolite\_manual}
\item
  Define the internal standard and/or alkanes\footnote{Also in
    conversion\_metabolite.csv; see below paragraph Standards}
\item
  Knit with parameter: \texttt{MTXQC\_ExperimentalSetup.Rmd}
\item
  Knit with parameter: \texttt{MTXQC\_part1.Rmd}
\item
  Knit with parameter: \texttt{MTXQC\_part2.Rmd}
\item
  If required - proceed with \texttt{MTXQC\_part3.Rmd} for
  ManualValidation
\end{enumerate}

\section{Input files}\label{input-files-1}

If you need an introduction about how to use metmax - have a look at the
separate documentation \texttt{Metmax\_intro}.

The chapter \ref{part4} \texttt{MTXQCvX\_part4} explains in detail how
to use this module to generate suitable input files.

\section{Annotation-file}\label{annotation-file-1}

The annotation file relate file names with experimental conditions or
specify quantification standards in your batch. Two columns -
\textbf{File and Type} - are obligatory and have to be present in the
annotation file. In the case of their absence
\texttt{MTXQCvX\_part1.Rmd} stops processing and shows an error message.

A quick way to generate an annotation file is described below:

\begin{enumerate}
\def\labelenumi{\arabic{enumi}.}
\tightlist
\item
  Copy all file names from a file of your choice
\item
  Paste \& transpose the content into a new Excel-File into column A
\item
  Call column A -\textgreater{} File
\item
  Optional: Remove any non-file name entry in this column
\item
  Add the column Type and specify each file either as \textbf{sample},
  \textbf{Q1\_diluation}, ,\textbf{addQ1\_dilution}\footnote{see for
    further details additionalQuant}
\item
  Add more columns specifying your experimental conditions, e.g.,
  Cellline and Treatment\footnote{optimal: two-three parameter, max:
    four parameter. Consider possible combinations, e.g.,
    HCT116-control, HCT116-BPTES}
\item
  Save the content as \texttt{csv-file} in the
  \texttt{psirm\_glucose/input/...}
\end{enumerate}

\section{Sample\_extracts-file}\label{sample_extracts-file-1}

The \texttt{sample\_extracts.csv} file is required in order to determine
automatically absolute quantities in the manner of pmol/1e+6 cells or
pmol/mg tissue in the \texttt{CalculationFileData.csv}.

This file requires two obligatory columns - \textbf{Extract\_vol} and
\textbf{Unit}\footnote{Define: count, mg or ul}. Please specify for each
experimental condition the amount of extracted cells (count), tissue
(mg) or volume of blood/plasme (ul) in the unit shown in the brackets.\\
The names of the columns of the experimental conditions have to match up
with the annotation file. Save the file in the folder
\texttt{psirm\_glucose/input/...}.

If the defined experimental conditions do not match up with the
annotation \texttt{MTXQCvX2\_part1.Rmd} exit data processing. A template
file is saved for review and usage at \texttt{inst/template\_files/...}

\section{\texorpdfstring{Update metabolite names in
\texttt{conversion\_metabolite.csv}}{Update metabolite names in conversion\_metabolite.csv}}\label{update-metabolite-names-in-conversion_metabolite.csv}

The file \texttt{conversion\_metabolite.csv}, saved in
\texttt{config\_mtx/}, serves as a kind of translational table. It
defines alternative version of metabolite library names that come in
handy to plot data using shorter metabolite names. This file is also
used to define settings and standard classifications. Detailed
information for each column of the file are shown here: REF

\subsection{Match your annotation with library
names}\label{match-your-annotation-with-library-names}

Prior the analysis you need to match the names of your intermediates
with the conversion\_metabolite.csv file. You need to update or add the
corresponding name for each intermediate in the column
\textbf{Metabolite\_manual}.

General suggestion for naming conventions in ChromaToF:
Metabolite\_Derivate, e.g., Lactic acid\_(2TMS). In case of the presence
of main- (MP) and byproducts (BP) use: Metabolite\_Derivate\_MP/BP,
e.g., Glucose\_(1MEOX)(5TMS)\_MP.

If you have annotated intermediates that are not included so far in this
table please follow the instructions how to extend
\texttt{conversion\_metabolite.csv}.REF

\subsection{Define your internal standards and
alkanes}\label{define-your-internal-standards-and-alkanes}

MTXQCvX2 allows the specification of project-specific internal
standards. Corresponding compounds have to be marked as an internal
standard in \texttt{conversion\_metabolite.csv} by adding the tag
\textbf{InternalStandard} in the column Standard.

If you check the box - InternalStandard in the parameter selection for
\texttt{MTXQCvX2\_part4.Rmd} the module searches in your input file for
peak areas of the defined standard and extracts the information. It also
generates the file \texttt{InternalStandard.csv} and stores it at
\texttt{psirm\_glucose/input/gc/...}.

In the same way alkanes are defined in
\texttt{conversion\_metabolite.csv}. Each alkane has to be flag tagged
with \textbf{Alk} in the column Standard. This gives you the opportunity
to implement customized mixtures of alkanes in order to determine the
retention index. \texttt{MTXQCvX\_part4.Rmd} recognises the flag tag and
generates \texttt{Alcane\_intensities.csv} based on your input file
containing peak areas and saves it in
\texttt{psirm\_glucose/input/gc/...}\footnote{It should be
  al\textbf{k}ane, I know, but Maui doesn't, unfortunately\ldots{}}.

The in-lab protocol considers nine alkanes from c10 to c36. Standard
annotation includes an hashtag, e.g., \#c10. If you use this annotation
even Metmax would be able to determine the retention index.

\chapter{MTXQCvX2\_init}\label{init}

MTXQCvX2\_init.Rmd - why and how to use it. Advantages of the project
folder.

\chapter{MTXQCvX\_experimentalSetup.Rmd}\label{ExpSetup}

\chapter{MTXQCvX\_part1.Rmd}\label{part1}

\chapter{MTXQCvX\_part2.Rmd}\label{part2}

\chapter{MTXQCvX\_part3.Rmd}\label{mtxqcvx_part3.rmd}

\chapter{MTXQCvX\_part4.Rmd - Metmax
parser}\label{mtxqcvx_part4.rmd---metmax-parser}

\section{This section explains \ldots{}}\label{this-section-explains}

\begin{itemize}
\tightlist
\item
  what MTXQCvX\_part4.Rmd does
\item
  how do input files need to look like
\item
  which files are generated
\item
  what the distinct checkboxes mean
\end{itemize}

This module provides the generation of suitable input files for MTXQCvX2
based on spreadsheet exported information by tools like metmax.

M

\section{Input files}\label{input-files-2}

\subsection[Quantification -
PeakAreas.csv]{\texorpdfstring{Quantification - PeakAreas.csv\footnote{Required
  for: all parameter, just not
  \texttt{calculation\ stable\ isotope\ incorporation}}}{Quantification - PeakAreas.csv}}\label{quantification---peakareas.csv}

In order to perform absolute quantification of

You need a file containing all extracted peak areas for each metabolite
and file\footnote{Tools/Options/Retention analysis, Parameter: Area}.
The header of metmax-extracted files looks like shown below (see table
1). Please, remember to delete the second header row, representing the
column loads for each file before saving as csv-file. Otherwise you end
up with weird imported dataframes in R. Quantification masses have to be
updated while processing in ChromaToF prior the export of the data e.g.,
with a reference search\footnote{See \texttt{vignette/ReferenceSearch}}
or using statistical compare. pSIRM experiments require the definition
of pTop5 masses\footnote{Extended list of quant masses considering
  isotope incorporation} instead of top5 masses in the reference in
order to take into account the shift of intensities induced by the
application of stable isotopes\footnote{Mandatory columns: name, mass,
  files}

\begin{tabular}{l|r|r|r|r|r|r}
\hline
name & mass & ri & row.load & file\_1 & file\_2 & file\_x\\
\hline
Lac & 219 & 1051 & 0.76 & 15423 & 135444 & 465486\\
\hline
Pyr & 174 & 1042 & 0.65 & 56978 & 46888 & 4354544\\
\hline
Cit & 273 & 1805 & 0.99 & 1326 & 23321 & 132121\\
\hline
\end{tabular}

MTXQCvX\_part4 takes care of the formatting and correct column names of
the peak areas file and saves it\footnote{\texttt{input/quant/quantMassAreasMatrix.csv}}.
MTXQCvX\_part4 generates also the file
PeakDensities-Chroma.csv\footnote{\texttt{input/gc/PeakDensities-Chroma.csv}},
in case you have selected the option to include sum of area
normalisation while knitting this module.

\subsection[Isotope incorporation - MIDs.csv]{\texorpdfstring{Isotope
incorporation - MIDs.csv\footnote{Required for
  \texttt{calculation\ isotope\ incorporation}}}{Isotope incorporation - MIDs.csv}}\label{isotope-incorporation---mids.csv}

In order to determine the incorporation of stable isotopes MTXQCvX2
requires as an input the mass isotopomer distributions (MIDs) for each
intermediate and measurement\footnote{Tools/Options/Isotope
  concentrator; Parameter: IntensityOfMass}. Fragments for each
intermediate have to be pre-defined in metmax at
Tools/Options/metabolite masses. They can be imported\footnote{\texttt{inst/template\_files/MetMax\_MIDs.txt}}
or manually specified each by each. An example of the metmax output is
shown in table 2. The output has to be saved as csv-file, including the
deletion of the partial row \texttt{column.load}, respectively\footnote{Mandatory
  columns: name, mass, files}.

\begin{tabular}{l|r|r|r|r|r|r}
\hline
name & mass & ri & row.load & file\_1 & file\_2 & file\_x\\
\hline
Lac & 219 & 1051 & 0.85 & 31026 & 5165829 & 5829\\
\hline
Lac & 220 & 1051 & 0.85 & 3607 & 662277 & 277\\
\hline
Lac & 221 & 1051 & 0.85 & 1222 & 111481 & 81\\
\hline
Lac & 222 & 1051 & 0.85 & 188 & 1003494 & 10023\\
\hline
Lac & 223 & 1051 & 0.85 & 0 & 33542 & 342\\
\hline
\end{tabular}

MTXQCvX\_part4 calculates the stable isotope incorporation and exports
DataMatrix.csv as well as pSIRM\_SpectraData.csv\footnote{\texttt{input/inc/DataMatrix}
  \& \texttt{pSIRM\_SpectraData.csv}}. The mathematics behind are
outlined in \citep{Pietzke2014}

\textbf{Important}: Extracted MIDs have to match with defined mass
couples for each metabolite in MTXQCvX2\footnote{\texttt{config\_mtx/incorpo\_calc\_masses.csv}}.
Please refer for more details to \texttt{vignettes/config\_mtx-files}.

\subsection[Derivatisation efficiency -
mz73.csv]{\texorpdfstring{Derivatisation efficiency - mz73.csv\footnote{Required
  for: \texttt{sum\ of\ area\ normalisation}}}{Derivatisation efficiency - mz73.csv}}\label{derivatisation-efficiency---mz73.csv}

The extraction of intensities for the ion \(m/z\) 73 works analogous to
the extraction of MIDs\footnote{Tools/Options/Isotope concentrator;
  Parameter: IntensityOfMass}. Mass ranges have to be defined for each
intermediate for the mass 73 by defining starting and end mass with 73.
MTXQCvX\_part4 generates the file MassSum-73.csv\footnote{\texttt{input/gc/MassSum-73.csv}}.
Check \texttt{inst\textbackslash{}template\_files\textbackslash{}} for
reference. Hopefully soon a new metmax button extracting specific
intensities across the batch.

\chapter{Configuration of MTXQCvX2}\label{config}

Herein explained are the customizable tables of the MTXQCvX2 universe.

\section{conversion\_metabolite.csv}\label{conversion_metabolite.csv}

\begin{Shaded}
\begin{Highlighting}[]
\NormalTok{data =}\StringTok{ }\KeywordTok{read.csv}\NormalTok{(}\StringTok{"config_mtx/conversion_metabolite.csv"}\NormalTok{, }\OtherTok{TRUE}\NormalTok{, }\DataTypeTok{sep =} \StringTok{";"}\NormalTok{)}
\KeywordTok{colnames}\NormalTok{(data)}
\end{Highlighting}
\end{Shaded}

\begin{verbatim}
##  [1] "Metabolite_manual" "Metabolite"        "Metabolite_short" 
##  [4] "Lettercode"        "Q1_value"          "Mass_Pos"         
##  [7] "SE_sel"            "Q_sel"             "nopsirm"          
## [10] "Standards"
\end{verbatim}

\section{Metabolic profile}\label{metabolic-profile}

\section{Defintion of intermediates}\label{defintion-of-intermediates}

\section{Declaration of standards}\label{declaration-of-standards}

\subsection{Internal standard}\label{internal-standard-1}

\subsection{Alkanes}\label{alkanes}

\subsection{quant1-values.csv}\label{quant1-values.csv}

\subsection{incorporation\_calc.csv}\label{incorporation_calc.csv}

\chapter{Protocols}\label{protocols}

\section{Sample extraction}\label{sample-extraction}

\subsection{Cell extracts}\label{cell-extracts}

Materials:

\begin{itemize}
\tightlist
\item
  cell culture dishes (10 cm)
\item
  50 \% MeOH plus 2 ug/ul cinnamic acid, ice-cold
\item
  chloroform
\item
  15 ml falcon tubes
\item
  cell lifter
\end{itemize}

Procedure:

\begin{itemize}
\tightlist
\item
  prepare cell culture dishes accordingly to your experimental
  conditions
\item
  discard media / buffer
\item
  add immediately 5 ml MeOH
\item
  detach cells using cell lifter
\item
  collect and transfer cell extract into 15 ml falcon
\item
  add 1 ml chloroform
\item
  incube for 30 - 60 min at cold temperature on rotary or thermo shaker
\item
  centrifuge at max speed for 10 min
\item
  collect polar and lipid phases
\item
  dry under vacuum
\end{itemize}

In order to generate technical backups:

\begin{itemize}
\tightlist
\item
  resuspend dried extracts in 600 ul 20 \% MeOH
\item
  shake at cold temperature on thermo shaker for 30 min
\item
  split volumes into equal parts in fresh eppendorf tubes
\item
  dry under vacuum
\end{itemize}

Suggested cell density: \(2-3e+6\) cells / extract.

\subsection{Tissue samples}\label{tissue-samples}

\subsection{Blood samples}\label{blood-samples}

\section{Sample derivatisation}\label{sample-derivatisation}

\section{GC-MS measurement}\label{gc-ms-measurement}

\section{Data processing}\label{data-processing}

\subsection{Processing for Maui
annotation}\label{processing-for-maui-annotation}

\subsubsection{Resampling}\label{resampling}

\subsubsection{Combo-export}\label{combo-export}

\subsubsection{Maui quantification
masses}\label{maui-quantification-masses}

\subsubsection{MAUI pSIRM}\label{maui-psirm}

\subsubsection{Maui exports}\label{mauiexport}

\begin{enumerate}
\def\labelenumi{\arabic{enumi}.}
\tightlist
\item
  Input-Folder: gc
\end{enumerate}

\begin{itemize}
\tightlist
\item
  Alcane\_intensities.csv (Diagnostics/Export Alcane intensities)
\item
  InternalStandard.csv \footnote{see more details below}
\item
  MassSum-73.csv (Diagnostics/QC Mass Sum Export for mass 73)
\item
  PeakDensities-Chroma.csv (Diagnostics/ExportPeakDensities)
\end{itemize}

\begin{enumerate}
\def\labelenumi{\arabic{enumi}.}
\setcounter{enumi}{1}
\tightlist
\item
  Input-Folder: quant
\end{enumerate}

\begin{itemize}
\tightlist
\item
  ManualQuantTable.tsv - \textbf{rename it} - e.g., e18125cz.tsv
  \footnote{Location:
    Maui-project/export/QM-AbsoluteQuantification/\ldots{}} \footnote{C://Users/User-name/MauiProjects/\ldots{}}
\item
  quantMassAreasMatrix.csv (Quantification export of samplesPeakGroups)
\end{itemize}

\begin{enumerate}
\def\labelenumi{\arabic{enumi}.}
\setcounter{enumi}{2}
\tightlist
\item
  Input-Folder: inc
\end{enumerate}

\begin{itemize}
\tightlist
\item
  DataMatrix.csv (Export \% Label of pSIRM-samplesPeakGroups)
\item
  pSIRM\_SpectraData.csv (pSIRM Spectra Export)\footnote{Requires the
    selection of Natural\_MIDs.txt}
\end{itemize}

\subsection{Processing for Metmax data
extraction}\label{processing-for-metmax-data-extraction}

\subsubsection{Resampling}\label{resampling-1}

\subsubsection{1D-basic}\label{d-basic}

\subsubsection{Reference search}\label{reference-search}

\subsubsection{Export for Metmax}\label{export-for-metmax}

\subsubsection{Data extraction}\label{data-extraction}

\chapter{Frequently Asked Questions}\label{FAQ}

\section{What are additional quantification
standards}\label{what-are-additional-quantification-standards}

\section{How do I extend
conversion\_metabolite.csv}\label{how-do-i-extend-conversion_metabolite.csv}

\bibliography{book.bib,packages.bib}


\end{document}
